% This is ''sig-alternate.tex'' V2.0 May 2012
% This file should be compiled with V2.5 of '\'sig-alternate.cls'' May 2012
%
% This example file demonstrates the use of the \'sig-alternate.cls'
% V2.5 LaTeX2e document class file. It is for those submitting
% articles to ACM Conference Proceedings WHO DO NOT WISH TO
% STRICTLY ADHERE TO THE SIGS (PUBS-BOARD-ENDORSED) STYLE.
% The \'sig-alternate.cls' file will produce a similar-looking,
% albeit, 'tighter' paper resulting in, invariably, fewer pages.

\documentclass{sig-alternate}
\usepackage{paralist}
\usepackage{url}

%
\def\sharedaffiliation{%
\end{tabular}
\begin{tabular}{c}}
%
\begin{document}
%
% --- Author Metadata here ---
\conferenceinfo{WIPSCE}{2015 London, UK}
\CopyrightYear{2015} % Allows default copyright year (20XX) to be over-ridden - IF NEED BE.
%\crdata{0-12345-67-8/90/01}  % Allows default copyright data (0-89791-88-6/97/05) to be over-ridden - IF NEED BE.
% --- End of Author Metadata ---

% working title!
\title{Introducing STEM to early years through child-led activities}

% alphabetical for the time being...
\numberofauthors{3}
    \author{
      \alignauthor Ana C. Calderon\\      
      \email{acalderon@cardiffmet.ac.uk}
%
      \alignauthor Tom Crick\\     
      \email{tcrick@cardiffmet.ac.uk}
%
      \alignauthor Catherine Tryfona\\     
      \email{ctryfona@cardiffmet.ac.uk}
% 
      \sharedaffiliation
      \affaddr{Department of Computing}  \\
      \affaddr{Cardiff Metropolitan University}   \\
      \affaddr{Cardiff, UK}
          }
%
\maketitle

\begin{abstract}
Increasingly we witness educational tools for early years through tcechnology. We  present recommendations for building future mobile application (with child-led engagement) to aid early years accomplish STEM activities, through computational thinking. The requirements were obtained from the results of a pilot study in which we investigated how assitance and reinforcement impacted a child's engagement. We found that the most engaging apps were those that allowed for minimum interference from adults and we found no difference in a child's engagement between visual and auditory reinforcements.
\end{abstract}

% A category with the (minimum) three required fields
\category{K.3.2}{Computers \& Education}{Computer and Information Science Education}[Computer Science Education]
\category{K.4.1}{Computers And Society}{Public Policy Issues}
\keywords{Early Years Education, Computational Thinking, Mobile Apps}

\section{Introduction}
There has been a surge in mobile technology for early years education (for the purposes of this work early years comprises 3-5 year olds). Several apps are constantly created to teach children specific aspects relevant to STEM, for instance pattern recognition, counting. Most STEM introducing sessions have, so far, been degined for teacher/adult-led activities (e.g. \cite{aronin2013},  \cite{abbas2014ontocog}). We argue that an adult-led activity might be required for children unfamiliar with the particular device (Ipad, IPhone, Android, etc.) but once that is no longer an obstacle, the key to developing a desire to learn STEM subjects in the futures is best achieved through child-led activities. We also give requirements for developing software, encouraging child-led activities, teaching aspects crucial to computational thinking. Child-led or "free play" activities \cite{bredekamp1987developmentally} consists of period in which the child is allowed freedom to choose activities that are a close match to their interests, rather than having a pre-prescribed session conducted by the adult/teacher. These have became common place in U.K. based preschools, and its place has already been established preschool learning \cite{DiChilvers}.
Computational thinking is being increasingly regarded as a fundamental skill for everyone in the 21st century \cite{Yadav2014} and as important a life skill as reading, writing and arithmetic \cite{Wing}.  Recently, there has been an increasing emphasis on improving the computational thinking skills in school children, particularly in K-12 in the US \cite{Settle2012} and Key Stages 2 and 3 in the UK \textbf{(Tom got a reference for this?  If not, I'll try to find one)}.  Whilst this is typically achieved through programming activities, there is increasing recognition that CT skills can be developed through ‘unplugged’ activities, including storytelling \cite{Thies2012}.  \cite{Lu2009} argue that programming should be the entrance to higher-level computer science and that those children who are exposed to CT in their formative education are likely to be better prepared for a CS curriculum.  Whilst there has been limited research in to the teaching of computational thinking specifically in early-years education, we argue that, given the potential for computational skills to be developed through play-based learning, CT skills development can be introduced as early on in formal education as the Early Years Foundation phase. 

\section{Pilot Study}
To decide requirements we ran a pilot study which comprised of a session with two parents and two children and we gave them a series of mobile apps to play with for the duration of one afternoon. The particular aspects we investigated were:
\begin{itemize}
\item \textbf{Reinforcement} Does the presence or absence of reinforcements for positive outcome in the app have an imapact with the child's engagement? We observed that both children played longer with STEM apps that contained reinforcement messages when the children did an activity correctly. There was no observed difference wether it was a visual reinforcement such as balloons or pleasent obejects displayed on-screen or a vocalized message "congratulations", "good job", etc.
\item \textbf{Assitance from adult} Does the intereference needed by adult counterpart impact the child's engagement? We observed that both children played longer with STEM apps that contained vocalized instructions and other mechanisms to minimize the adult's involvement. 
\end{itemize}
Both have been positively answered by the pilot study, but this will be tested further by a full study. 
We will now describe the list of recommendations and explain the result further.


\section{Recommendations}
A summary of recomendations for future system is given, this is based on sociological and pegagogical research as well as considerations of human -comuter interaction and design. 

\section*{Embedded Reinforcement}
There are several potential ways of motivating early year pupils to learn, preschool teachers have specific strategies targeting the specific ages. A good comparisson of varying methods, especifically used by preschool teachers can be found in \cite{hanley2009influencing}. Reinforcement through mobile application is crucial to maitain engagement of the child in the activity - display of baloons, confetti and a short spoken "congratulations" message are good examples of this.
\subsection*{Collaborative Desgin} 
Child-led activites can be collaboratively shared with others in a playgroup or nursery setting, and it is important to design learning applications that encourage children to share their excitement in new discoveries amongst their peers. Hence, we recommend that these apps be developed for collaborative usage, as oppose to having sessions of individual children with individual devices, not communicating with each other. We recommend a setting in which the children are at physical proximity to each other,  either a tabletop design with several different children conducting different activities, or individual devices with apps that enable a "friend" button that children can send/receive a simple symbol, e.g. a happy face smiley so other kids can approach the child and see what they are investigating. 
\subsection*{Assistance}
Adult supervision and micromanagement of the activities should not be required, however since proficiency of ICT is variable and since children of that age are notorious for displaying frustration with ICT  usagae when incapable of achieving their goals, we suggest that a child should have the ability to grab an adult for support within a few seconds of encoutering any difficulty.
\subsection*{Monitoring}
We recommend that these apps have the ability to track the child's progress in elarning specific aspects of Computational thinking and that these be broadcasted to carers/teachers in an appropriate means. 

\section{Conclusion and future work}
We have presented recommendations for future software systems that aid in the teaching of computational thinking to early years children (3-5 year olds).
cite BritishHCI.
Immediate future work will comprise a full investigation into the adequacies of our requirements, which will be an expansion of our pilot study.
% bib
\bibliographystyle{abbrv}
\bibliography{wipsce2015}


\end{document}
