% This is ''sig-alternate.tex'' V2.0 May 2012
% This file should be compiled with V2.5 of '\'sig-alternate.cls'' May 2012
%
% This example file demonstrates the use of the \'sig-alternate.cls'
% V2.5 LaTeX2e document class file. It is for those submitting
% articles to ACM Conference Proceedings WHO DO NOT WISH TO
% STRICTLY ADHERE TO THE SIGS (PUBS-BOARD-ENDORSED) STYLE.
% The \'sig-alternate.cls' file will produce a similar-looking,
% albeit, 'tighter' paper resulting in, invariably, fewer pages.

\documentclass{sig-alternate}
\usepackage{paralist}
\usepackage{url}

%
\def\sharedaffiliation{%
\end{tabular}
\begin{tabular}{c}}
%
\begin{document}
%
% --- Author Metadata here ---
\conferenceinfo{WIPSCE}{2015 London, UK}
\CopyrightYear{2015} % Allows default copyright year (20XX) to be over-ridden - IF NEED BE.
%\crdata{0-12345-67-8/90/01}  % Allows default copyright data (0-89791-88-6/97/05) to be over-ridden - IF NEED BE.
% --- End of Author Metadata ---

% working title!
\title{Using Interface Design to Develop\\Computational Thinking Skills}
%Interface Design and Its Role in Computational Thinking
%Thinking About Design, Thinking About Computing?
%The Interface Between Design and Computational Thinking?


\numberofauthors{2}
\author{
% 1st. author
\alignauthor
Ana C. Calderon\\
\affaddr{Department of Computing}\\
\affaddr{Cardiff Metropolitan University, UK}\\
\affaddr{acalderon@cardiffmet.ac.uk}
% 2nd. author
\alignauthor
Tom Crick\\
\affaddr{Department of Computing}\\
\affaddr{Cardiff Metropolitan University, UK}\\
\affaddr{tcrick@cardiffmet.ac.uk}\\
}

\maketitle

\begin{abstract}
Human-computer interaction is an established sub-discipline of
computer science. There is, however, a gap in its teaching through
computational thinking, we present the first step to methodologies for
systematically introducing HCI to pupils from an early age, in the
same computational thinking style to other sub-disciplines of computer
science that have already been considered for the past decade.
\end{abstract}

% A category with the (minimum) three required fields
\category{K.3.2}{Computers \& Education}{Computer and Information Science Education}[Computer Science Education]
\category{K.4.1}{Computers And Society}{Public Policy Issues}
\keywords{Computer Science Education; High School; Teachers}

\section{Introduction}
Computational thinking is increasingly seen for significance in the teaching of foundations of computer science as well for its ability to aid in deveoping skills across a range of subjects.
HCI is a well established area of significance to computer science, but also it is a multi-disciplinary discipline which, we argue, has aspects of relevance to computational thinking. Interactivity and design has been somewhat neglected during the growth of computational thinking and there is also a gap in the teaching of computing (cite Kent people). Some proof-of-concept solution methodologies are detailed, targeted at particular aspects of HCI, presented in the form of sessions accessible to children of primary school age.




 
\section{Suggested sessions}
We now give methodologies, in the form of activity-based learning sessions, to teach elements crucial to HCI to secondary school students. These are ideas are influence by \cite{normanDesign}, \cite{rogers2011interaction} and \cite{shneiderman1986designing}.
\subsection{Session A}
The first session we describe is intended to cover negative transfer as well as the principle of affordance, and we break down the explanation not by the flow of the session, but by the intended learning outcome, this is intended to facilitate also the explanation of the effects making the presentation accessible outside the HCI community. The session was devised with the use of a "doll house" specifically designed in a self-contradictory manner. The students are given clues to navigate around the house and their goal is to find a pebble. To find the hidden pebble, the students must follow instructions and activities specified in the cards. This follows similar patterns as a popular game show, and the reason is to help the students view this as a fun activity rather than a typical learning session (via traditional classroom environments).

\subsubsection*{Principle of affordance}
Affordances are the prduct of agents and their enviromnet \cite{gibson1977theory}, gien any agent-environment combination a affordance may or may not exist. If one does the agent needs to be aware of it, however for most HCI researchers when affordance is mentioned it is typically assumed that the agent is aware of it,  calls this a perceived affordance \cite{norman1999affordance}.

The very first instruction given to students is to simply enter the house, the entrance consists of a front door, porch, both doors have a door knob, the first door must be pushed whereas the second door needs to be pulled. Students are likely to try to turn the knob the first time and push the second. At this stage in the session, an explanation is due, on how past experiences can affect learning of new tasks. In addition most students are likely to attempt to turn the knob, which makes it an excellent point to teach students about the principle of affordance. 


\subsubsection*{Explaining transfer effects applied to desgin methods in HCI}
In a nutshell negative transfer\cite{Lunchin, Pan2010, Woltz} is a term behavioural psychologist use to describe how new task learning can be negatively affected by knowledge of similar or related tasks.
This has consequences to design of interactive designs \cite{waern1993varieties} For example,\cite{Besnard2005105} uses a simulated experiment (based on real industrial work environments) to support a hypothesis of how interface changes are less likely to cause accidents when it limits changes to former interaction patterns.

Amongst the instructions, pupils are asked to fill up a watering can in the kitchen, water the plants, return to the house and clean the same watering can in the downstairs guest bathroom. The first tap used must be turned clockwise in order to let out water, conversely the second tap must be turned anti-clockwise for the same goal to be achieved. This is the second intentional opportunity during the activity, in which teachers can explain transfer effects and start a discussion bout its importance to design.

\subsection{Session B}
The second session is based on a similar activity as published in \cite{fellows2005} where students are given cooking ingredients lists to transform into symbols.

\subsubsection*{Design for recognition}
 
As the above section demonstrate, learning how to achieve certain tasks will have consequences to learning new tasks. As humans we are constantly learning new skills and it is thus essential to design intuitive interactive devises. The following two sessions are aimed at highlighting the importance of designing to minimise the amount of learning required.

For this, pupils are placed into two groups, the first group is given three lists: ingredients, portions and instructions. They must find a way to abstract away from the textual information and draw icons and symbols that will aid the second group of pupils put together a recipe. The list is small, "chocolate, milk, butter" and the portions are simple "a little", "a lot", "a big piece", "a medium piece", the instructions are also simple "start mixing", "stop mixing", "start pouring", etc. The students should enjoy themselves as they try to discover whether how close to the original recipe the second group's recipe is. After the session is complete the teacher should emphasise their learning of difficult but important it is to design interfaces with icons that are meaningful. Examples of solutions to text abstraction can be found in \cite{fig1} (representing small, medium and large piece).


\begin{figure}
  \includegraphics[width=\columnwidth]{Portions.jpg}
  \caption{Example of representing portions via text.}\label{fig1}
\end{figure}

\subsubsection*{Iterative Desgin}
Iterative design: After determining the users, tasks, and empirical measurements to include, perform the following iterative design steps:
Design the user interface
Test
Analyze results
Repeat
Repeat the iterative design process until a sensible, user-friendly interface is created.

In addition, this session can be structured to teach other desgin-relevant methodologies to children, namely iterative desgin. This is achieved by allowing the pupils second and third attempts at their representation of the text given to them (we expect more than three attmepts would be frustrating for the pupil, and three are enough to illustrate the principle of iteration in desgin). The pupils must be told that the iteration is intended to improve the design and should be repeated until a good and sensible set of symbols is achieved (as would happen with an interface with programmers). This should be followed by a short formative session on iterative design, hihglighting the main concepts in a language acessible to the particular age group, for instance:
\begin{enumerate}
\item What is the target audience of your interactive software? What is the goal of your software? What empirical testing can be done to check for accuracy in achieving the tasks post creation of your interface?
\item Design your interface 
\item Test your interface
\item Interpret results
\item Iterate (repeat)
\item End when satisfied with resulting interface
\end{enumerate}


The sessions described are intended as examples of principles relevant to HCI, this is the begining of a list of potentially several principles that are importatnt to HCI, we mean for this sessions to be seen as illustrative examples rather than containing a comprehensive list of principles.


%%%
\section{Conclusions}
We have presented a methodology for introducing good-design principles and theories relevant to human-computer interaction to young children. This is exemplified with two sessions developed encompassing computational thinking.We hypothesise that HCI can be embedded as part of computational thinking skills in young pupils, and that these will have a positive impact across their future learning, not just for computing. Furthermore, we reiterate the wider societal benefits of developing broad computational skills, both as baseline digital competencies to ensure a digitally engaged citizenry, as well as high value skills for the economies of the future.
Immediate future work will consist in an implementation of the sessions with pupils to investigate their feasibility and how they 

% bib
\bibliographystyle{abbrv}
\bibliography{wipsce2015}


\section*{To Do}

In no particular order...

\begin{itemize}
\item Add stuff here...
\end{itemize}

\subsection*{References to use}
\begin{itemize}
\item
General CAS
citations~\cite{crick+sentance:2011,brown-et-al-sigcse2012,brown-et-al-toce2014}.

\item
Teachers, CPD and
NoE~\cite{sentance-et-al-wipsce2012,sentance-et-al:2013,sentance-et-al:2014}.

\item
Welsh Government reports/policy:
\begin{itemize}
\item
ICT Review~\cite{welshictreview:2013}
\item
Graham Report on STEM~\cite{STEMreview:2014}
\item
Donaldson Report~\cite{Donaldson:2015}
\item
Furlong Report~\cite{Furlong:2015}
\end{itemize}

\item
Misc Reports
\begin{itemize}
\item
Nesta report~\cite{NESTA:2015}

\end{itemize}
\end{itemize}

\end{document}
